\documentclass[a4paper, 11pt]{article}
\usepackage[utf8]{inputenc}
\usepackage[french]{babel}

\usepackage[T1]{fontenc}
\usepackage{amsmath}
\usepackage{amsfonts}
\usepackage{amssymb}
\usepackage{authblk}

\usepackage{hyperref}
\usepackage{listings, graphicx}
\usepackage{color, hyperref}
\hypersetup{
	colorlinks=true,
	linktoc=all,
	linkcolor=black,
}
\usepackage{lmodern}  % for bold teletype font
\usepackage{amsmath}  % for \hookrightarrow
\usepackage{xcolor}   % for \textcolor

\usepackage{titling}

\lstset{literate=
  {á}{{\'a}}1 {é}{{\'e}}1 {í}{{\'i}}1 {ó}{{\'o}}1 {ú}{{\'u}}1
  {Á}{{\'A}}1 {É}{{\'E}}1 {Í}{{\'I}}1 {Ó}{{\'O}}1 {Ú}{{\'U}}1
  {à}{{\`a}}1 {è}{{\`e}}1 {ì}{{\`i}}1 {ò}{{\`o}}1 {ù}{{\`u}}1
  {À}{{\`A}}1 {È}{{\'E}}1 {Ì}{{\`I}}1 {Ò}{{\`O}}1 {Ù}{{\`U}}1
  {ä}{{\"a}}1 {ë}{{\"e}}1 {ï}{{\"i}}1 {ö}{{\"o}}1 {ü}{{\"u}}1
  {Ä}{{\"A}}1 {Ë}{{\"E}}1 {Ï}{{\"I}}1 {Ö}{{\"O}}1 {Ü}{{\"U}}1
  {â}{{\^a}}1 {ê}{{\^e}}1 {î}{{\^i}}1 {ô}{{\^o}}1 {û}{{\^u}}1
  {Â}{{\^A}}1 {Ê}{{\^E}}1 {Î}{{\^I}}1 {Ô}{{\^O}}1 {Û}{{\^U}}1
  {œ}{{\oe}}1 {Œ}{{\OE}}1 {æ}{{\ae}}1 {Æ}{{\AE}}1 {ß}{{\ss}}1
  {ű}{{\H{u}}}1 {Ű}{{\H{U}}}1 {ő}{{\H{o}}}1 {Ő}{{\H{O}}}1
  {ç}{{\c c}}1 {Ç}{{\c C}}1 {ø}{{\o}}1 {å}{{\r a}}1 {Å}{{\r A}}1
  {€}{{\euro}}1 {£}{{\pounds}}1 {«}{{\guillemotleft}}1
  {»}{{\guillemotright}}1 {ñ}{{\~n}}1 {Ñ}{{\~N}}1 {¿}{{?`}}1
}

\lstset{
  basicstyle=\ttfamily,
  columns=fullflexible,
  frame=single,
  breaklines=true,
  postbreak=\mbox{\textcolor{red}{$\hookrightarrow$}\space},
}
%\pretitle{\begin{center}\fontesize{16pt}{16pt}\selectfont}
%\posttitle{\par\end{center}\vskip 3em}
\font\myfont=cmr12 at 42pt
\title{\myfont {Robots programmables}}
\author{\huge {COTREZ Léo}\and \huge {ORNIACKI Thomas}\and \\Université Paris-VIII, Saint-Denis, France}

\begin{document}
\maketitle

\newpage
\tableofcontents

\newpage
\section{Procédures mise en \oe uvre}
\subsection{Les problèmatiques}
\subsubsection{Titre de la sous-sous-section}
\paragraph{Titre du paragraphe}
\subparagraph{Titre du sous-paragraphe}

\newpage
\section{Listing du programme}
\subsection{Les problèmatiques}
\subsection{Les fonctions de game.c}

\newpage
\section{Mode d'emploi}
\subsection{Comment ouvrir un terminal}
\subsubsection{}

\newpage
\section{Traces d'utilisation}
\subsection{Sreen shoots}

\newpage
\section{Code complet}
\subsection{Les déclarations du jeu}
%\lstinputlisting[language=c]{../game.h}

\newpage
\subsection{Le jeu}
%\lstinputlisting[language=c]{../game.c}

\newpage
\subsection{Le client}
%\lstinputlisting[language=c]{../client.c}

\newpage
\subsection{Le serveur}
%\lstinputlisting[language=c]{../server.c}

\newpage
\subsection{Les fonctions minimalistes}
%\lstinputlisting[language=c]{../fct_mini.c}

\newpage
\subsection{L'interpréteur}
%\lstinputlisting[language=c]{../interpreteur.c}

\end{document}