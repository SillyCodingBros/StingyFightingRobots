\documentclass[a4paper, 11pt]{report}
\usepackage[utf8]{inputenc}
\usepackage[french]{babel}

\usepackage[T1]{fontenc}
\usepackage{amsmath}
\usepackage{amsfonts}
\usepackage{amssymb}
\usepackage{authblk}

\usepackage{hyperref}
\usepackage{listings, graphicx}
\usepackage{color, hyperref}
\hypersetup{
	colorlinks=true,
	linktoc=all,
	linkcolor=blue,
}
\usepackage{lmodern}  % for bold teletype font
\usepackage{amsmath}  % for \hookrightarrow
\usepackage{xcolor}   % for \textcolor

\usepackage{titling}



\lstset{
  basicstyle=\ttfamily,
  columns=fullflexible,
  frame=single,
  breaklines=true,
  postbreak=\mbox{\textcolor{red}{$\hookrightarrow$}\space},
}
%\pretitle{\begin{center}\fontesize{16pt}{16pt}\selectfont}
%\posttitle{\par\end{center}\vskip 3em}
\font\myfont=cmr12 at 42pt
\title{\myfont {Robots programmables}}
\author{\huge {COTREZ Léo}\and \huge {ORNIACKI Thomas}\and \\Université Paris-VIII, Saint-Denis, France}
\date{}
\begin{document}
\maketitle

\tableofcontents
\newpage
\chapter{Procédures mise en \oe uvre}
\section{Les problèmatiques}
\subsection{Titre de la sous-section}
\subsubsection{Titre de la sous-sous-section}
\paragraph{Titre du paragraphe}
\subparagraph{Titre du sous-paragraphe}
\chapter{Listing du programme}
\section{Les fonctions de game.c}
\chapter{Mode d'emploi}
\section{Comment ouvrir un terminal}
\chapter{Traces d'utilisation}
\section{Sreen shoots}
\chapter{Code complet}
\section{truc.c}
\end{document}